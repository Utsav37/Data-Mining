%%%%%%%%%%%%%%%%%%%%%%%%%%%%%%%%%%%%%%%%%
% Stylish Article
% LaTeX Template
% Version 2.1 (1/10/15)
%
% This template has been downloaded from:
% http://www.LaTeXTemplates.com
%
% Original author:
% Mathias Legrand (legrand.mathias@gmail.com) 
% With extensive modifications by:
% Vel (vel@latextemplates.com)
%
% License:
% CC BY-NC-SA 3.0 (http://creativecommons.org/licenses/by-nc-sa/3.0/)
%
%%%%%%%%%%%%%%%%%%%%%%%%%%%%%%%%%%%%%%%%%

%----------------------------------------------------------------------------------------
%	PACKAGES AND OTHER DOCUMENT CONFIGURATIONS
%----------------------------------------------------------------------------------------

\documentclass[fleqn,10pt]{SelfArx} % Document font size and equations flushed left

\usepackage[english]{babel} % Specify a different language here - english by default
\usepackage{lipsum} % Required to insert dummy text. To be removed otherwise
\setlength{\columnsep}{0.55cm} % Distance between the two columns of text
\setlength{\fboxrule}{0.75pt} % Width of the border around the abstract
\definecolor{color1}{RGB}{0,0,90} % Color of the article title and sections
\definecolor{color2}{RGB}{0,20,20} % Color of the boxes behind the abstract and headings
\usepackage{hyperref} % Required for hyperlinks
\hypersetup{hidelinks,colorlinks,breaklinks=true,urlcolor=color2,citecolor=color1,linkcolor=color1,bookmarksopen=false,pdftitle={Title},pdfauthor={Author}}
\usepackage{algorithm}
\usepackage{algorithmic}
\usepackage{multirow}
\usepackage{amsmath}
\usepackage{amssymb}
\def\Plus{\texttt{+}}	
\def\Minus{\texttt{-}} 
\usepackage{amsfonts}
\usepackage{alltt}
\usepackage{url}
\usepackage{relsize}


%----------------------------------------------------------------------------------------
%	ARTICLE INFORMATION
%----------------------------------------------------------------------------------------

\JournalInfo{Computer Science\\School of Informatics, Computing, and Engineering\\Indiana University, Bloomington, IN, USA} % Journal information
\Archive{Datamining B565 Fall 2017}  % Additional notes (e.g. copyright, DOI, review/research article)

\PaperTitle{Kaggle Competitions:\\ \ Author Identification\\ \ Statoil/C-CORE Iceberg Classifier Challenge} % Article title

\Authors{Name\textsuperscript{1}*, Another Name\textsuperscript{2}} % Authors
\affiliation{\textsuperscript{1}\textit{Computer Science, School of Informatics, Computing, and Engineering, Indiana University, Bloomington, IN, USA}} % Author affiliation
\affiliation{\textsuperscript{2}\textit{Computer Science, School of Informatics, Computing, and Engineering, Indiana University, Bloomington, IN, USA}} % Author affiliation
\affiliation{*\textbf{Corresponding author}: name@indiana.edu} % Corresponding author

\Keywords{Keyword1 --- Keyword2 --- Keyword3} % Keywords - if you don't want any simply remove all the text between the curly brackets
\newcommand{\keywordname}{Keywords} % Defines the keywords heading name
\usepackage{abstract}

%----------------------------------------------------------------------------------------
%	Executive Summary
%----------------------------------------------------------------------------------------

\Abstract{\textbf{Executive Summary} Briefly explain the kaggle competition and datamining as a solution.  Briefly explain each problem, the solution, and results. }


\begin{document}

\renewcommand{\abstractname}{}  
\renewcommand{\absnamepos}{} 

\flushbottom % Makes all text pages the same height
\maketitle % Print the title and abstract box
\tableofcontents % Print the contents section
\thispagestyle{empty} % Removes page numbering from the first page

%----------------------------------------------------------------------------------------
%	ARTICLE CONTENTS
%----------------------------------------------------------------------------------------

\section{Introduction}

\begin{itemize}[noitemsep] %use noitem to make more compact
\item Briefly, but more completely than in the Executive Summary, explain the kaggle competition -- how to get to it, \textit{etc.}.
\item   Discuss datamining abstractly and how it fits as a solution to the kaggle competition.
\item Briefly, but more completely than in the Executive Summary, explain the two problems in two different subsections.
\end{itemize}
\subsection{Author Identification}
What is the problem to be solved?  What is the data? How is goodness quantified?  This should not be too technical, but we can say, for example, given three authors $A = \{a_1, a_2, a_3\}$ and selections of their individiually corresponding works $S_{a_1}, S_{a_2}, S_{a_3}$,  we are constructing a probability  mass function $f$ over $A$ that is applied to a text $t$ written by one of the authors, but unlabled:
\begin{equation}
f(A|t,S) = \{p_{a_1}, p_{a_2}, p_{a_3}\}
\end{equation}

A text is simply a passage from one of the author's works:
\begin{quote}
Once upon a midnight dreary, while I pondered, weak and weary, Over many a quaint and curious volume of forgotten lore—While I nodded, nearly napping, suddenly there came a tapping, As of some one gently rapping, rapping at my chamber door.“’Tis some visitor,” I muttered, “tapping at my chamber door— Only this and nothing more.”
\end{quote}
from Edgar Allen Poe's, \textit{The Raven}.  You must find out whether it's only prose -- and how much of the original structure is maintained.  Likely you will not take in text directly -- so this must be described.
\subsection{Statoi/C-CORE Iceberg Classifier Challenge}

  Use footnotes sparingly, but you must use at least two in this document\footnote{A footnote provides some ancillary information that's somewhat misplaced in the text, \textit{i.e.}, the reader can skip it without missing any critical information.  On the other hand, a dedicated reader appreciates the extra information.  A footnote might clarify information in an informal way.  Although possible, putting mathematics in a footnote is a bit odd.}.
  
\lipsum[3]

\section{Datamining}
\begin{itemize}[noitemsep]
\item What is datamining?
\item What does it yield?
\item What are the general steps?
\item What is clustering \textit{vs.} classfication?
\item What is a loss function?
\end{itemize}
\subsection{Data Preprossing}
\begin{itemize}[noitemsep]
\item What are the steps?
\item What is the general load (time, space, \$) for preprocessing
\item What challenges does each step present?
\end{itemize}
\subsection{Mining, Interpretation, and Action}

\begin{itemize}[noitemsep]
\item Briefly discuss the top 10 algorithms.
\item Does datamining tell us what to do?
\item What are some new types of problems in datamining?
\end{itemize}
%------------------------------------------------
%
%  Author Identication
%   
%------------------------------------------------
\section{Author Idenfication: Full Problem Description}
\begin{itemize}[noitemsep]
\item Define problem
\item Formally describe problem--inputs, outputs, training and testing method
\end{itemize}

\begin{figure*}[ht]\centering % Using \begin{figure*} makes the figure take up the entire width of the page
\includegraphics[width=\linewidth]{view}
\caption{You can span in the width of the page if you need to.}
\label{fig:view}
\end{figure*}

\lipsum[4] % Dummy text

\begin{equation}
\cos^3 \theta =\frac{1}{4}\cos\theta+\frac{3}{4}\cos 3\theta
\label{eq:refname2}
\end{equation}

\lipsum[5] % Dummy text


\subsection{Data Analysis}
\begin{itemize}[noitemsep]
\item Describe the data in full detail--from its raw form to the transformation
\item Provide summary statistics and relationships
\end{itemize}

\lipsum[6] % Dummy text

\subsection{Methods}
\begin{itemize}[noitemsep]
\item Discuss the algorithm you've chosen, \textit{e.g.}, why you chose it
\item Provide some background material on your method that shows you are well-acquianted with it
\item Challenges to your method
\item What software and hardware did you use, packages, \textit{etc.}
\item Final structure of data after preprocessing
\item Present training and testing as some combination of text and visualizations
\end{itemize}
\begin{figure}[ht]\centering
\includegraphics[width=\linewidth]{results}
\caption{Any illustration should first says what it is.  Then describe the axes.  What is important in the graph.  There should be a title too.}
\label{fig:results}
\end{figure}
Algorithm\,\ref{alg:1} shows an extended $k$-means.

\begin{algorithm}
\caption{This is a caption for the algorithm.  $k$-means* over $\Delta$}
\label{alg:1}
\begin{algorithmic}[1]
\STATE{\ {\bf INPUT} data $\Delta$, blocks k, distance $\textbf{d}:\Delta^2 \rightarrow \Re_{\geq 0}$}
\STATE{\ {\bf OUTPUT} centroids $C_1,\ldots,C_l$}
\STATE \%\% assume that a centroid is a pair $(v,X)$
\STATE \%\% $v\in \Re^m$ and (a possibly empty) $ X\subseteq \Delta$
\STATE \%\% heap $ H \subseteq \Delta$
\STATE{\ randomly construct k centroids $\mathbf{\mathsf{C}}^0 = \{C_1^0, C_2^0,\ldots,C_k^0\}$}
\STATE{\ $i \leftarrow 0$ }
\STATE \%\% $\Delta_{HE}$ represents HE data, $\Delta = \Delta_{HE} +  \Delta_{LE}$
\STATE{\ $\Delta_{HE} \leftarrow \Delta$}
\REPEAT
\FOR {$\mathbf{x} \in \Delta_{HE}$}
\FOR { $C_j^i \in \mathbf{\mathsf{C^i}}$}
\STATE \%\% assign data to centroid/heap that is nearest
\STATE \%\% $\sigma \Rightarrow d$
\STATE $C_j^i.H.insert(\mathbf{x},d)$, where min$\{ \textbf{d}(\mathbf{x}, C_j^i.v)\}$
\ENDFOR
\ENDFOR
\STATE{\ $\Delta^\prime \leftarrow \emptyset$}
\FOR {$C_j^i \in \mathbf{\mathsf{C^i}}$}
\STATE \%\% recalculate centroid as average of over $C.H$
\STATE $C_j^{i+1}.v \leftarrow  \sum\nolimits_{\mathbf{x} \in C_j^i.X} (\mathbf{x}/|C_j^i.X|) $
\STATE{\ $\Delta^\prime \leftarrow C_j^i.H.flush(\sigma)$}
\STATE{\ $ \mathbf{\mathsf{C^{i+1}}} \xleftarrow{\cup} \{ C_j^{i+1}\} $}
\ENDFOR
\STATE{\ $i \leftarrow i+1$ }
\STATE{\ $\Delta_{HE} \leftarrow \Delta^\prime$}
\UNTIL{threshold on $\mathbf{\mathsf{C^{i-1}}}$}
\end{algorithmic}
\end{algorithm}


Reference to Figure \ref{fig:results}.

%------------------------------------------------

\subsection{Results}
\begin{itemize}[noitemsep]
\item Dispassionately describe your results both quantified and qualified
\item Do you deem this successful
\item What do the results suggest
\item What were challenges
\end{itemize}
Remember, we're interested in the journey, so simply because an approach failed doesn't mean failure if you discuss the failure!
\subsection{Summary and Future Work}
\begin{itemize}[noitemsep]
\item Briefly summarize project and outcome
\item What would you do differently in the future?
\end{itemize}

\lipsum[10] % Dummy text
\section{Iceberg: Full Problem Description}


\lipsum[11] % Dummy text
We used $10$-fold...
\begin{table}[hbt]
\caption{Training and Test Results}
\centering
\begin{tabular}{llr}
\toprule
\multicolumn{2}{c}{Name} \\
\cmidrule(r){1-2}
 \textsf{Data} & \textsf{Description} & \textsf{$Sum$} \\
\midrule
1 & ``Hello...'' & $7.5$ \\
2 & ``Goodbye...'' & $2$ \\
\bottomrule
\end{tabular}
\label{tab:label}
\end{table}


\subsubsection{Citations and Subsubsection}
This level should be used sparingly, but can provide a sense to the reader of the outline and, therefore, relationship of information.  You'll need to have at least 10 citations.  Here is an example.  This work was began by Micky Mouse\cite{vbku10}.  Here's a way to provide bullet points:

\begin{description}
\item[Word] Definition
\item[Concept] Explanation
\item[Idea] Text
\end{description}

%------------------------------------------------
\phantomsection
\section*{Acknowledgments} % The \section*{} command stops section numbering
Put people, Grants, \textit{etc.} that helped contribute to the success and completion of the work.  Be generous.
\addcontentsline{toc}{section}{Acknowledgments} % Adds this section to the table of contents



%----------------------------------------------------------------------------------------
%	REFERENCE LIST
%----------------------------------------------------------------------------------------
\phantomsection
\bibliographystyle{unsrt}
\bibliography{sample}

%----------------------------------------------------------------------------------------

\end{document}